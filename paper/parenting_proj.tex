% Template for Cogsci submission with R Markdown

% Stuff changed from original Markdown PLOS Template
\documentclass[10pt, letterpaper]{article}

\usepackage{cogsci}
\usepackage{pslatex}
\usepackage{float}
\usepackage{caption}

% amsmath package, useful for mathematical formulas
\usepackage{amsmath}

% amssymb package, useful for mathematical symbols
\usepackage{amssymb}

% hyperref package, useful for hyperlinks
\usepackage{hyperref}

% graphicx package, useful for including eps and pdf graphics
% include graphics with the command \includegraphics
\usepackage{graphicx}

% Sweave(-like)
\usepackage{fancyvrb}
\DefineVerbatimEnvironment{Sinput}{Verbatim}{fontshape=sl}
\DefineVerbatimEnvironment{Soutput}{Verbatim}{}
\DefineVerbatimEnvironment{Scode}{Verbatim}{fontshape=sl}
\newenvironment{Schunk}{}{}
\DefineVerbatimEnvironment{Code}{Verbatim}{}
\DefineVerbatimEnvironment{CodeInput}{Verbatim}{fontshape=sl}
\DefineVerbatimEnvironment{CodeOutput}{Verbatim}{}
\newenvironment{CodeChunk}{}{}

% cite package, to clean up citations in the main text. Do not remove.
\usepackage{cite}

\usepackage{color}

% Use doublespacing - comment out for single spacing
%\usepackage{setspace}
%\doublespacing


% % Text layout
% \topmargin 0.0cm
% \oddsidemargin 0.5cm
% \evensidemargin 0.5cm
% \textwidth 16cm
% \textheight 21cm

\title{A performance model for early word learning}


\author{{\large \bf Emily Hembacher} \\ \texttt{ehembach@stanford.edu} \\ Department of Psychology \\ Stanford University \And {\large \bf Michael C. Frank} \\ \texttt{mcfrank@stanford.edu} \\ Department of Psychology \\ Stanford University}

\begin{document}

\maketitle

\begin{abstract}
\ldots{}

\textbf{Keywords:}
Parenting attitudes; implicit theories
\end{abstract}

Child development research is constantly generating information that can
be brought to bear on best-practices for parenting. For example,
research on children's learning has demonstrated that pedagogy can
improve learning in some contexts and limit it in others, suggesting
that allowing children to play freely and explore is critical for
learning (Bonawitz et al., 2011; Buchsbaum, Gopnik, Griffiths, \&
Shafto, 2011). Likewise, a great deal of research has demonstrated the
importance of engaging young children in elaborative conversations for
language development and future academic success (Hart \& Risley, 1995;
Hoff, 2003; Huttenlocher, Waterfall, Vasilyeva, Vevea, \& Hedges, 2010).
A fundamental challenge we face is how to communicate the results of
such scientific inquiry to a diverse public in a way that maximizes
uptake and improves people's daily and long-term decision making.

One critical parameter that may moderate the uptake of new information
is parents' implicit theories about child development and parenting.
Research on implicit theories has found that people's lay theories in
different domains (which may or may not be explicitly articulated)
organize the processing of new information. For example, people with an
entity theory of personality tend to interpret people's behaviors as
stemming from fixed personality traits rather than situational factors
such as needs, goals, or emotional states (Dweck, Chiu, \& Hong, 1995).
Similarly, there is evidence that people use lay theories about emotion
in order to infer people's current emotions from contextual cues (Ong,
Zaki, \& Goodman, 2015).

To provide an example of how lay theories might operate in the domain of
parenting, parents who believe that children are not capable of learning
before they are able to speak may be slower to uptake information
suggesting that free play is helpful for causal learning among infants.
Indeed, there is evidence supporting the notion that parents' behaviors
are guided by implicit lay theories about child development, which vary
by SES and across cultures. For example, cross-cultural studies have
found profound differences in how parents interact with infants;
Richman, Miller \& LeVine (1992) found that mothers in the Gusii
community of Kenya primarily engaged with their children to soothe them
when upset, but did not often speak to them with the goal of engaging or
stimulating them, as did Caucasian parents in the United States. The
authors attribute this to cultural conventions stemming from the belief
that there is no purpose in speaking to infants as they will not
understand what is being said (Richman, Miller \& LeVine, 1992; LeVine,
2004).

There are also important differences in how parents within western
cultures interact with their children. Numerous studies have identified
SES disparities in the amount that parents talk to their children, which
predicts children's language and academic outcomes (refs). In an effort
to identify the source of this disparity, Rowe (2008) discovered that
parents' knowledge of child development (as indexed by their scores on
the Knowledge of Infant Development Inventory; KIDI) predicted their
child-directed language, with more knowledgeable parents speaking to
their children more even when controlling for the amount of speech
directed at another adult. Although this study examined parents
knowledge, and not their lay theories per se, this can be taken as
evidence that parents are approaching parenting with different priors
with regard to their roles as parents.

There are other examples of parenting theories? dimensions on which
parents differ; Lareau's (2003) theory of ``concerned cultivation''
suggests that higher SES parents are more likely to view their child's
development as a project that requires a great deal of coordination in
the form of activities and learning experiences, while lower SES parents
are more likely to view their job as keeping their children safe from
harm, with the assumption that they will naturally thrive if given
independence. Furthermore, there have been hundreds of studies based on
Baumrind's (1971) framework that identifies parents as authoritative,
authoritarian, or permissive, based on their levels of responsiveness
and control in their interactions with their children. Thus, parents'
approach to parenting appear to vary in predictable ways based on their
knowledge and perceptions surrounding children's learning and
development. A further question is whether these lay theories also
moderate the uptake of new information. Previous studies of parenting
styles have generally relied on observation of parent-child interactions
or self-report of specific activities and behaviors. To our knowledge
there is not an existing measure of parents' more general attitudes
about parenting and child development, which might predict the uptake of
interventions. The present study had two purposes: first, to establish a
self-report scale to capture adults' lay theories about child
development and parenting, and second, to test the hypothesis that
uptake will be greater for information consistent with people's prior
lay theories about child development.

In order to establish a new measure of parenting attitudes, we followed
a structured plan based on psychometric best practices (Clark \& Watson,
1995; Furr, 2010; Simms, 2008). We generated items corresponding to
three hypothesized latent theories about parenting: the Early Learning
theory corresponds to a view of children's early learning that is
consistent with contemporary child development research, and includes
the idea that young children can teach themselves by exploring and
playing (refs). The Affection and Attachment theory captures the notion
that close parent-child relationships are important for development, and
includes the ideas that parents should talk to their children about
their emotions and that children are not spoiled by too much affection
(refs). The Rules and Respect theory corresponds to the idea that
parents' primary role is to enforce rules and encourage behavior control
(refs). We generated items based on a review of the literature on
parenting attitudes, and conducted psychometric analyses on iterative
samples of respondents. After establishing a final set of scale items,
we conducted an experiment to test whether scores on the three subscales
would predict people's uptake of new information. For this purpose, we
had participants read a popular press article on one of two parenting
topics: an article arguing that free play is beneficial to children's
learning, or an article arguing that time-outs are harmful to children's
development. We operationalized uptake as accurate recall and
generalization of the central messages of the target articles. We
predicted that people's scores on the Early Learning subscale would
moderate uptake of the ``free play'' article, and that scores on the
Rules and Respect scale would moderate uptake of the ``time-outs''
article. (some description of findings that don't exist yet).

\section{Acknowledgements}\label{acknowledgements}

This work supported by a gift from Kinedu, Inc. Thanks to members of the
Language and Cognition Lab at Stanford for helpful discussion.

\section{References}\label{references}

\small

\end{document}
