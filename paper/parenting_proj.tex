% Template for Cogsci submission with R Markdown

% Stuff changed from original Markdown PLOS Template
\documentclass[10pt, letterpaper]{article}

\usepackage{cogsci}
\usepackage{pslatex}
\usepackage{float}
\usepackage{caption}

% amsmath package, useful for mathematical formulas
\usepackage{amsmath}

% amssymb package, useful for mathematical symbols
\usepackage{amssymb}

% hyperref package, useful for hyperlinks
\usepackage{hyperref}

% graphicx package, useful for including eps and pdf graphics
% include graphics with the command \includegraphics
\usepackage{graphicx}

% Sweave(-like)
\usepackage{fancyvrb}
\DefineVerbatimEnvironment{Sinput}{Verbatim}{fontshape=sl}
\DefineVerbatimEnvironment{Soutput}{Verbatim}{}
\DefineVerbatimEnvironment{Scode}{Verbatim}{fontshape=sl}
\newenvironment{Schunk}{}{}
\DefineVerbatimEnvironment{Code}{Verbatim}{}
\DefineVerbatimEnvironment{CodeInput}{Verbatim}{fontshape=sl}
\DefineVerbatimEnvironment{CodeOutput}{Verbatim}{}
\newenvironment{CodeChunk}{}{}

% cite package, to clean up citations in the main text. Do not remove.
\usepackage{cite}

\usepackage{color}

% Use doublespacing - comment out for single spacing
%\usepackage{setspace}
%\doublespacing


% % Text layout
% \topmargin 0.0cm
% \oddsidemargin 0.5cm
% \evensidemargin 0.5cm
% \textwidth 16cm
% \textheight 21cm

\title{A new measure of parents' lay theories of parenting and child
development}


\author{{\large \bf Emily Hembacher} \\ \texttt{ehembach@stanford.edu} \\ Department of Psychology \\ Stanford University \And {\large \bf Michael C. Frank} \\ \texttt{mcfrank@stanford.edu} \\ Department of Psychology \\ Stanford University}

\begin{document}

\maketitle

\begin{abstract}
\ldots{}

\textbf{Keywords:}
Parenting attitudes; implicit theories
\end{abstract}

Child development research is constantly generating information that can
be brought to bear on best-practices for parenting. For example,
research on children's learning has demonstrated that pedagogy can
improve learning in some contexts and limit it in others, suggesting
that allowing children to play freely and explore is critical for
learning (Bonawitz et al., 2011; Buchsbaum, Gopnik, Griffiths, \&
Shafto, 2011). Likewise, a great deal of research has demonstrated the
importance of engaging young children in elaborative conversations for
language development and future academic success (Hart \& Risley, 1995;
Hoff, 2003; Huttenlocher, Waterfall, Vasilyeva, Vevea, \& Hedges, 2010).
A fundamental challenge we face is how to communicate the results of
such scientific inquiry to a diverse public in a way that maximizes
uptake and improves people's daily and long-term decision making.

One critical parameter that may mediate parenting behavior is parents'
implicit lay theories about child development and parenting. Lay
theories reflect the core beliefs that people hold in different domains,
which may or may not be explicitly articulated, but organize the
processing of new information and decision-making (Dweck \& Legget,
1988; Ong, Zaki, \& Goodman, 2015). For example, people with an entity
theory of personality tend to interpret people's behaviors as stemming
from fixed personality traits rather than situational factors such as
needs, goals, or emotional states (Dweck, Chiu, \& Hong, 1995). There
are two reasons to focus on parents' lay theories. First, parents' lay
theories might be an important explanatory factor for many of the
behaviors parents engage in with their child. For example, a parent who
believes that building a strong emotional bond with their baby is one of
the most important goals of parenting might have more physical contact
with their child than a parent who does not hold this theory. Secondly,
parents' lay theories may moderate the uptake of new information about
parenting. It is well-established that people more easily encode new
information that is consistent with an existing schema or mental model
they hold (Bransford \& Johnson, 1972). In addition, previous research
has found that interventions on public health beliefs are more
successful when they take into account people's existing belief
structures in the domain (Kumar et al. 2015).

There is some evidence supporting the notion that parents' behaviors are
mediated by implicit lay theories about child development, which vary by
SES and across cultures. For example, cross-cultural studies have found
profound differences in how parents interact with infants; Richman,
Miller \& LeVine (1992) found that mothers in the Gusii community of
Kenya primarily engaged with their children to soothe them when upset,
but did not often speak to them with the goal of engaging or stimulating
them, as did Caucasian parents in the United States. The authors
attribute this to cultural conventions stemming from the belief that
there is no purpose in speaking to infants as they will not understand
what is being said (Richman, Miller \& LeVine, 1992; LeVine, 2004).

There are also important differences in how parents within western
cultures interact with their children. Numerous studies have identified
SES disparities in the amount that parents talk to their children, which
predicts children's language and academic outcomes (Hoff, 2003;
Huttenlocher et al. 2002). In an effort to identify the source of this
disparity, Rowe (2008) discovered that parents' knowledge of child
development (as indexed by their scores on the Knowledge of Infant
Development Inventory; KIDI) predicted their child-directed language,
with more knowledgeable parents speaking to their children more even
when controlling for the amount of speech directed at another adult.
Although this study examined parents' knowledge, and not their lay
theories per se, it provides evidence that people's domain knowledge has
real consequences for their interactions with their children.

There are other examples of parenting beliefs on which parents differ;
Lareau's (2003) theory of ``concerned cultivation'' suggests that higher
SES parents are more likely to view their child's development as a
project that requires a great deal of coordination in the form of
activities and learning experiences, while lower SES parents are more
likely to view their job as keeping their children safe from harm, with
the assumption that they will naturally thrive if given independence.
There have also been hundreds of studies based on Baumrind's (1971)
framework that identifies parents as authoritative, authoritarian, or
permissive, based on their levels of responsiveness and control in their
interactions with their children. Thus, parents' approaches to parenting
appear to vary in predictable ways based on their knowledge and
perceptions about children's learning and development.

Although these previous studies provide preliminary evidence that
parents' beliefs about parenting and child development affect their
parenting behaviors, no previous research has attempted to identify the
underlying theories that might organize their behavior and
decision-making. Previous research has generally relied on observation
of parent-child interactions or self-report of specific activities and
behaviors. To our knowledge there is not an existing measure of parents'
more general attitudes about parenting and child development, which
might drive behavior and predict the uptake of interventions. To address
this gap, the present work establishes a self-report scale that captures
adults' lay theories about child development and parenting. We generated
a questionnaire measuring the degree to which parents endorse three
potential lay theories: a ``rules and respect'' theory, an ``affection
and attachment'' theory, and an ``early learning'' theory. As an initial
test of the external validity of the questionnaire, we conducted an
experiment to investigate whether parents' scores on the theory
subscales would differentially predict their uptake of parenting
information presented via a popular press article about children's early
learning from free play (Gopnik, 2011). We found that higher scores on
the ``early learning'' subscale, but not the ``rules and respect''
subscale, predicted recall and generalization from the target article
(about free play) but not a control article that was unrelated to child
development. Thus, parents' lay theories about parenting and child
development as measured by our questionnaire may be a meaningful factor
in parents' behavior and information uptake.

\section{Scale Construction}\label{scale-construction}

In order to establish a new measure of parenting attitudes, we followed
a structured plan based on psychometric best practices (Clark \& Watson,
1995; Furr, 2010; Simms, 2008). We generated items corresponding to
three hypothesized latent theories about parenting: the Early Learning
theory corresponds to a view of children's early learning that is
consistent with contemporary child development research, and includes
the idea that young children can teach themselves by exploring and
playing. The Affection and Attachment theory captures the notion that
close parent-child relationships are important for development, and
includes the ideas that parents should talk to their children about
their emotions and that children are not spoiled by too much affection.
The Rules and Respect theory corresponds to the idea that parents'
primary role is to enforce rules and encourage behavior control. We
generated items based on a review of the literature on parenting
attitudes, and conducted psychometric analyses on iterative samples of
respondents.

In an initial phase of scale construction, we generated 42 statements
that described attitudes consistent with one of three potential implicit
theories about parenting: Active Learning (12 items; e.g., ``Children
can learn about things like good and bad behavior from an early age''),
Affection and Attachment (10 items; e.g., ``It's important for a baby to
have a strong bond with mom''), and Rules and Respect (20 items; e.g.,
``It is very important that children learn to respect adults, such as
parents and teachers''). These statements were generated based on a
literature review of parenting attitudes and behaviors. The Affection
and Attachment and Rules and Respect subscales are related theoretically
to the Authoritative and Authoritarian dimensions of Baumrind's (1971)
parenting framework, as well as theories of attachment parenting (Jones,
Cassidy, \& Shaver, 2015), but aim to assess beliefs about parenting
rather than overt behaviors. The Early Learning subscale aimed to assess
the extent to which adults believe that it is important to help infants
and toddlers learn through play and conversation.

The initial 42-item scale was administered to 250 adults on Amazon's
Mechanical Turk. Participants used a 7-point Likert scale to report the
degree to which they agreed with each statement from 0 (Do not Agree) to
6 (Strongly Agree). Chronbach's alphas for the three subscales were .86
(Active Learning), .81 (Affection and Attachment), and .74 (Rules and
Respect). We then conducted Exploratory Factor Analysis (EFA) to assess
the dimensionality of the scale. Based on a parallel analysis (Horn,
1965) we retained 5 factors in this initial model. We subsequently
dropped any items that had factor loadings less than .40 on the relevant
factor, as well as any items that had factor loadings greater than .40
onto another factor. Items were also dropped if analyses revealed that
Chronbach's alpha would be increased by dropping the item. Additional
items were dropped such that there were 6 items in each subscale. Some
items were re-worded such that half of the items in each subscale were
negatively worded to avoid response sets (Simms, 2008).

The revised questionnaire was administered to a second group of 250
adults on Amazon's Mechanical Turk. For this sample, Chronbach's alphas
were .76 (Active Learning), .75 (Affection and Attachment), and .69
(Rules and Respect). Because analysis of the previous sample identified
5 factors instead of the hypothesized 3, we again conducted EFA. This
time, the parallel analysis identified 3 factors as predicted. The
subscale items were roughly grouped according to a priori subscales,
although some items from the Affection and Attachment subscale load onto
both the Affection and Attachment and Active Learning subscale factors.

\section{Experiment 1}\label{experiment-1}

We next conducted an experiment to test whether scores on the three
subscales would predict people's uptake of new information, as an
initial test of the external validity of the scales. For this purpose,
we had participants read two popular press articles: an article arguing
that free play is beneficial to children's learning (Gopnik, 2011), and
an article about the language of smell (Yong, 2015). We operationalized
uptake as accurate recall and generalization of the central message of
the target article, and recall of the control article. We predicted that
if people's subscale scores reflect coherent lay theories, they should
differentially moderate uptake of the two articles. Specifically, we
predicted that scores on the Early Learning subscale would be positively
related to recall and generalization of the target article, but not
recall of the control article. We predicted that scores on the Rules and
Respect subscale would not predict uptake of either article. We excluded
scores on the Affection and Attachment subscale from our analyses, since
they are not orthogonal to scores on the Early Learning subscale.

\subsection{Methods}\label{methods}

\subsubsection{Participants}\label{participants}

Participants were 250 adults recruited from Amazon's Mechanical Turk.

\subsubsection{Procedure}\label{procedure}

Map on question short forms so that we can use these instead.

Clean up labels.

Merge. Recode uptake answers by accuracy.

Plot demographic info.

\begin{CodeChunk}
\begin{CodeOutput}

   0    1    2    3    4    5    6 
  77  132  212  524  731 1017 1807 
\end{CodeOutput}
\begin{CodeOutput}

         0          1          2          3          4          5 
0.01711111 0.02933333 0.04711111 0.11644444 0.16244444 0.22600000 
         6 
0.40155556 
\end{CodeOutput}
\end{CodeChunk}

\begin{CodeChunk}

\includegraphics{figs/unnamed-chunk-8-1} \end{CodeChunk}

\begin{CodeChunk}

\includegraphics{figs/unnamed-chunk-11-1} \end{CodeChunk}

\begin{CodeChunk}
\begin{CodeOutput}
[1] 48
\end{CodeOutput}
\end{CodeChunk}

\begin{CodeChunk}

\includegraphics{figs/unnamed-chunk-13-1} \end{CodeChunk}

\textbackslash{}begin\{CodeChunk\} \textbackslash{}begin\{CodeOutput\}
Generalized linear mixed model fit by maximum likelihood (Laplace
Approximation) {[}glmerMod{]} Family: binomial ( logit ) Formula:
correct \textasciitilde{} question\_type * rules\_respect +
question\_type * active\_learning +\\ (1 \textbar{} workerid) + (1
\textbar{} q\_num) Data: filter(d.reg, !workerid \%in\% exclude)

\begin{verbatim}
 AIC      BIC   logLik deviance df.resid 
\end{verbatim}

39100.8 39198.8 -19539.4 39078.8 54529

Scaled residuals: Min 1Q Median 3Q Max -6.3483 0.0554 0.2232 0.4042
8.7323

Random effects: Groups Name Variance Std.Dev. workerid (Intercept)
2.7895 1.6702\\ q\_num (Intercept) 0.6011 0.7753\\Number of obs: 54540,
groups: workerid, 202; q\_num, 15

Fixed effects: Estimate Std. Error z value (Intercept) 1.11009 0.36004
3.083 question\_typetarget\_generalize 1.56790 0.48045 3.263
question\_typetarget\_recall 1.27814 0.48109 2.657 rules\_respect
-0.10182 0.21071 -0.483 active\_learning 0.27433 0.22355 1.227
question\_typetarget\_generalize:rules\_respect -0.05937 0.06306 -0.942
question\_typetarget\_recall:rules\_respect 0.32354 0.05755 5.622
question\_typetarget\_generalize:active\_learning 0.61527 0.05663 10.864
question\_typetarget\_recall:active\_learning 0.48608 0.05420 8.968
Pr(\textgreater{}\textbar{}z\textbar{})\\(Intercept) 0.00205 **
question\_typetarget\_generalize 0.00110 ** question\_typetarget\_recall
0.00789 ** rules\_respect 0.62895\\active\_learning
0.21978\\question\_typetarget\_generalize:rules\_respect
0.34643\\question\_typetarget\_recall:rules\_respect 1.88e-08
\textbf{\emph{ question\_typetarget\_generalize:active\_learning
\textless{} 2e-16 }} question\_typetarget\_recall:active\_learning
\textless{} 2e-16 *** --- Signif. codes: 0 `\emph{\textbf{' 0.001 '}'
0.01 '}' 0.05 `.' 0.1 `' 1

Correlation of Fixed Effects: (Intr) qstn\_typtrgt\_g qstn\_typtrgt\_r
rls\_rs actv\_l qstn\_typtrgt\_g -0.655\\qstn\_typtrgt\_r -0.657
0.487\\rules\_rspct -0.001 0.002 0.000\\activ\_lrnng -0.067 0.003 0.000
-0.254\\qstn\_typtrgt\_gnrlz:r\_ 0.001 -0.003 -0.001 -0.088 0.025
qstn\_typtrgt\_rcll:r\_ 0.002 -0.001 0.000 -0.099 0.028
qstn\_typtrgt\_gnrlz:c\_ 0.005 0.001 -0.002 0.029 -0.093
qstn\_typtrgt\_rcll:c\_ 0.005 -0.003 -0.003 0.032 -0.099
qstn\_typtrgt\_gnrlz:r\_ qstn\_typtrgt\_rcll:r\_
qstn\_typtrgt\_g\\qstn\_typtrgt\_r\\rules\_rspct\\activ\_lrnng\\qstn\_typtrgt\_gnrlz:r\_\\qstn\_typtrgt\_rcll:r\_
0.401\\qstn\_typtrgt\_gnrlz:c\_ -0.386 -0.141\\qstn\_typtrgt\_rcll:c\_
-0.133 -0.335\\ qstn\_typtrgt\_gnrlz:c\_
qstn\_typtrgt\_g\\qstn\_typtrgt\_r\\rules\_rspct\\activ\_lrnng\\qstn\_typtrgt\_gnrlz:r\_\\qstn\_typtrgt\_rcll:r\_\\qstn\_typtrgt\_gnrlz:c\_\\qstn\_typtrgt\_rcll:c\_
0.446\\\textbackslash{}end\{CodeOutput\}
\textbackslash{}end\{CodeChunk\}

\subsection{Results}\label{results}

\subsection{Discussion}\label{discussion}

\section{Acknowledgements}\label{acknowledgements}

This work supported by a gift from Kinedu, Inc. Thanks to members of the
Language and Cognition Lab at Stanford for helpful discussion.

\section{References}\label{references}

\small

\end{document}
